\documentclass[13pt, a4paper, twoside]{article}
\usepackage[utf8]{inputenc}
\usepackage{geometry}
\usepackage[czech]{babel}
\usepackage{fancyhdr}
\usepackage{mathtools}
\usepackage{hyperref}
\usepackage{float}
\usepackage{setspace}
\usepackage{ragged2e}
\usepackage{multicol}
\usepackage{graphicx}
\DeclareUnicodeCharacter{0301}{ś}
\geometry{legalpaper, margin=1.05in}
\pagestyle{fancy}
\lhead{\Large Předčítač (1995)}
\rhead{\large Bernhard Schlink}
\begin{document}
\onehalfspacing \large

Hanna bojovala dále. Přiznávala, co bylo pravda, a odmítala, co pravda nebylo. Odmítala se stále zoufalejší naléhavostí. Nikdy nebyla hlučná. Ale už sama intenzita, se kterou hovořila, soud rušila.

Nakonec to vzdala. Mluvila jen, když byla tázána, odpovídala krátce, uboze, mnohdy roztěkaně. Jakoby na důkaz, že to vzdala, zůstávala nyní, když hovořila, sedět. Předsedající soudce, který jí na počátku přelíčení několikrát řekl, že nemusí stát, ale klidně zůstat sedět, vzal tuto změnu s údivem na vědomí. Ke konci přelíčení jsem několikrát nabyl dojmu, jako by soud byl vším unaven, chtěl celou záležitost už už uzavřít, nesoustředil se, byl myšlenkami jinde, po dlouhých týdnech minulosti konečné v přítomnosti.

I já toho měl dost. Nemohl jsem to však nechat jen tak. Přelíčení pro mě neskončilo, ale začínalo. Byl jsem divákem a najednou se stal účastníkem, spoluhráčem a spolurozhodčím. Tuto novou roli jsem ani nehledal, ani nevolil, ale najednou tady byla, ať už jsem si to přál nebo ne, něco dělal nebo se choval naprosto pasivně.

Něco dělat - a přitom šlo jen o jedno. Mohl jsem zajít k předsedajícímu soudci a říci mu, že Hanna je negramotná. Že nebyla hlavní aktérkou a viníkem, kterou z ní chtěly ostatní obžalované udělat. Že se její chování během procesu nevyznačovalo zvláštní nepoučitelností, nepřístupností a neomaleností, ale že vyplývalo z nedostatečné znalosti obžaloby a rukopisu knihy a nejspíše i z absence jakéhokoliv smyslu pro strategii a taktiku. Že se při své obhajobě značně poškozovala. Že je vinna, ale ne tolik, jak se mohlo zdát.

Možná bych předsedajícího soudce nepřesvědčil. Přiměl bych jej však k zamyšlení a novému zkoumání. Nakonec by se ukázalo, že mám pravdu, a Hanna by sice byla potrestána, ale méně. Musela by do vězení, ale dostala by se z něj dříve, dříve by byla na svobodě - nebylo to snad to, zač Hanna bojovala?

Ano, bojovala za to, ale nebyla za úspěch ochotna zaplatit cenu odhalení své negramotnosti. Rovněž by nechtěla, abych to, co chtěla utajit, prodal za pár let ve vězení. Takový obchod mohla udělat sama, ale neudělala, a tudíž nechtěla. Její postoj ji stál za několik let vězení.

Stálo to však skutečně za to? Co z takového vylhaného chování, které ji jenom svazovalo, ochromovalo a nenechávalo prostor pro vlastní rozhodování, mohla mít? Energii, kterou na udržování své životní lži vynakládala, mohla již dávno věnovat tomu, aby se naučila číst a psát.

Snažil jsem se hovořit o tomto problému alespoň se svými přáteli. Představ si, že se někdo vrhá vědomě do záhuby, a ty ho můžeš zachránit - zachráníš ho? Představ si operaci pacienta, který bere drogy, jež se nesnesou s anestetiky, ale on se stydí, že drogy bere, a nechce to anesteziologům říci
- promluvil bys s nimi? Představ si soudní přelíčení a obžalovaného, který bude potrestán, když se nepřizná, že je levák, a nemohl tudíž spáchat trestný čin, který byl vykonán pravou rukou, ale stydí se, že je levák - řekneš soudci, jak se věci mají? Představ si homosexuála, který se díky své orientaci nemohl dopustit určitého trestného činu, ale stydí se za svou homosexualitu. Nejde o to, zda se má stydět za to, že je levák nebo homosexuál jenom si představ, že se obžalovaný stydí.
 

\end{document}