\documentclass[13pt, a4paper, twoside]{article}
\usepackage[utf8]{inputenc}
\usepackage{geometry}
\usepackage[czech]{babel}
\usepackage{fancyhdr}
\usepackage{mathtools}
\usepackage{hyperref}
\usepackage{float}
\usepackage{setspace}
\usepackage{ragged2e}
\usepackage{multicol}
\usepackage{graphicx}
\DeclareUnicodeCharacter{0301}{ś}
\geometry{legalpaper, margin=1.05in}
\pagestyle{fancy}
\lhead{\Large Swannova láska (1913)}
\rhead{\large Marcel Proust (1871-1922)}
\begin{document}
\onehalfspacing \large
Swann si však říkal, že bude-li dávat Odettě najevo (když ji bude
vyhledávat až po večeři), že má také jiné zájmy, jimž dává přednost před
její společností, nebude zalíbení, které v něm Odette nalézá, tak rychle
přesyceno. A na druhé straně ho neskonale víc než Odettina krása lákal
půvab mladé dělnice, svěží a kypré jako růže, do níž se zamiloval a s níž
rád trávil počátek večera, když si byl jist, že se později sejde s Odettou. Z
téhož důvodu nikdy nepřipustil, aby se pro něj Odette zastavila a vzala ho s
sebou k Verdurinovým. Mladá dělnice na něj čekávala blízko jeho bytu na
rohu, který jeho kočí Rémi znal, přisedla ke Swannovi a zůstávala v jeho
náruči, dokud kočár nezastavil před domem Verdurinových. Když vešel,
paní Verdurinová ukázala na růže, které jí ráno poslal, řekla: „Zlobím se,“ a
pokynula mu na místo vedle Odetty; pianista jim oběma zatím hrál
Vinteuilovu větu, jakousi píseň jejich lásky. Začínala výdrží houslových
tremol, která znějí několik taktů sama a zabírají celý první plán, pak se
náhle jakoby vzdálí, a jako když Pieter de Hooch prohlubuje své obrazy v
pozadí rozdílnou barvou a hebkostí vnikajícího světla v úzkém rámu
pootevřených dveří, podobně se objevovala ona tančící, pastorální, vsunutá,
epizodní větička, patřící jinému světu. Plynula v prostých a nesmrtelných
vlnách, rozdávajíc tu a tam dary svého půvabu se stejným nevystižitelným
úsměvem; Swann se však domníval, že v něm teď poznává rozčarování.
Zdálo se, že ta věta zná marnost štěstí, k němuž ukazuje cestu.
Ve svém
lehkém půvabu měla cosi završeného jako lhostejnost nastupující po lítosti.
Avšak na tom mu nezáleželo, zajímala ho ne tak sama o sobě — v tom, co
mohla znamenat pro skladatele, který, když ji psal, vůbec nevěděl o něm ani
o Odettě, a pro všechny, kdo ji během staletí uslyší —, ale spíš jako zástava,
jako památka jeho lásky, která i Verdurinovým a malému pianistovi
připomíná Odettu a zároveň jeho, která je spojuje; šel tak daleko, že když
ho o to Odette z rozmaru požádala, vzdal se úmyslu dát si od některého
umělce zahrát celou sonátu, z níž ani nadále neznal víc než tuto pasáž. „Nač
potřebujete to ostatní?“ ptala se ho. „Náš kousek je tohle.“

\noindent\makebox[\linewidth]{\rule{\paperwidth}{0.4pt}}

\section*{Neumělecký text}
Na začátku vztahu jsme omámené hormony a zaslepené růžovými brýlemi. Pod nadvládou oxytocinu máme pocit, že jsme potkaly osudového muže, kterému bychom odpustily všechny hříchy světa. Přesto bychom měly rozeznat, zda za nás hovoří láska, nebo chtíč. Pokud totiž hledáte dlouhodobý vztah, pak byste se měla vyhnout randění, které má kořeny právě ve chtíči.

„Chtíč je pocit, který vychází z fyzické touhy. Mimo vášeň má vztah poháněný chtíčem velmi málo podstaty. Lidé se totiž primárně snaží uspokojit vlastní potřeby,“ říká klinický psycholog Steve Sultanoff.

Naopak, pokud hledáte otevřený vztah, je důležité včas rozpoznat, zda chování partnera nevychází z lásky a zamilovanosti. „Na rozdíl od chtíče není láska majetnická. Když někoho milujete, berete ohled také na jeho zájmy a potřeby, zatímco touha je více o soustředění na vaše vlastní touhy,“ vysvětluje rozdíl Sultanoff.

\hspace*{\fill} \emph{zdroj: novinky.cz}

\end{document}