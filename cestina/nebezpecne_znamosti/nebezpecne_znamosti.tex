\documentclass[13pt, a4paper, twoside]{article}
\usepackage[utf8]{inputenc}
\usepackage{geometry}
\usepackage[czech]{babel}
\usepackage{fancyhdr}
\usepackage{mathtools}
\usepackage{hyperref}
\usepackage{float}
\usepackage{setspace}
\usepackage{ragged2e}
\usepackage{multicol}
\usepackage{graphicx}
\geometry{legalpaper, margin=0.5in}
\begin{document}
\onehalfspacing \large
Využil jsem prvního Dancenyho dopisu jeho krásce, a když jsem ji smluveným signálem upozornil, použil jsem veškeré své dovednosti nikoliv na to, abych jí ho předal, ale abych naopak k tomu neměl možnost: předstíral jsem, že sdílím netrpělivost, kterou jsem takto vyvolal, a když jsem způsobil zlo, naznačil jsem nápravu.

Ta mladá osoba bydlí v pokoji, jehož jedny dveře vedou na chodbu, ale matinka si od nich vzala klíč, jako by něco tušila. Šlo jen o to se toho klíče zmocnit. Nic snazšího než to provést. Žádal jsem jen, abych ho dostal na dvě hodiny, a zaručil jsem se, že dodám jeden podobný. Pak by se všechno, dopisování, rozhovory, noční schůzky, stalo pohodlné a bezpečné: ale věřila byste to? Nesmělé dítko se polekalo a odmítlo. Jiný člověk by z toho byl zoufalý; já v tom shledal příležitost pro ještě pikantnější potěšení. Napsal jsem Dancenymu a stěžoval si na to odmítnutí, což jsem udělal tak dobře, že náš pomatenec neustal, dokud si nevyžádal, ba nevynutil na své bázlivé milence, že přistoupí na mou žádost a zcela se podřídí mému rozhodnutí.

Musím přiznat, že se mi tato výměna úloh docela zamlouvala, když ten mladý muž učinil pro mne, co předpokládal, že učiním já pro něho. Tato představa cenu tohoto dobrodružství v mých očích zdvojnásobila: a tak sotva jsem dostal ten cenný klíč, spěchal jsem ho použít. Bylo to předešlé noci.

Když jsem se přesvědčil, že je na zámku klid, vypravil jsem se, vyzbrojen zacloněnou svítilnou a v oděvu, jaký odpovídal hodině a jaký vyžadovaly okolnosti, na první návštěvu k Vaší žačce. Dal jsem vše připravit (a to jí samou), abych mohl vstoupit nehlučně. Spala prvním spánkem a navíc spánkem jejího věku, takže jsem došel až k posteli, aniž se probudila.

Nejprve jsem byl v pokušení dál pokračovat a pokusit se zahrát sen, bál jsem se však překvapení a z toho vzniklého hluku, a proto jsem raději opatrně spící krásku probudil, čímž se mi skutečně podařilo předejít výkřiku, jehož jsem se obával.

Když jsem upokojil její počáteční obavy, dovolil jsem si několik smělostí, protože jsem přece nepřišel si popovídat. V klášteře jí zřejmě dobře nevysvětlili, jak rozmanitým nebezpečenstvím je vystavena nesmělá nevinnost a co vše musí chránit, aby nebyla zaskočena: protože soustředila veškerou pozornost a veškeré síly, aby se ubránila polibku, což byl jen falešný útok, vše ostatní zůstalo bez obrany. Není přece možné toho nevyužít! Změnil jsem proto směr a okamžitě jsem zaujal postavení. V tu chvíli jsme si oba mysleli, že jsme ztraceni: holčička, celá vyplašená, chtěla upřímně křičet, slzy naštěstí její výkřik udusily. Vrhla se rovněž na šňůru zvonku, ale já dokázal ruku včas zadržet.

»Co chcete dělat?« řekl jsem jí pak. »Chcete se navždy zničit? Co mi na tom sejde, když někdo přijde! Koho přesvědčíte, že jsem sem nepřišel s vaším svolením? Kdo jiný než vy by mi mohl poskytnout možnost sem vstoupit? A jste ochotna prozradit účel, jemuž má sloužit ten klíč, který jsem dostal od vás, který jsem mohl získat jedině vaším prostřednictvím?« Ten krátký proslov neuklidnil ani bol, ani hněv, přivodil však podřízení. Nevím, zda mu můj tón dodal výřečnosti. Rozhodně ho nevylepšila žádná gesta. Jedné ruky jsem používal pro násilí, druhé pro milování, což by mohl nějaký řečník v takovém postavení předstírat eleganci? Dovedete-li si však toto postavení dobře představit, uznáte zajisté, že bylo naopak příznivé pro útok. Ale já přece vůbec ničemu nerozumím, a jak říkáte, i ta nejprostší žena, i chovanka z kláštera mě dokáže vést jako dítě.

\noindent\makebox[\linewidth]{\rule{\paperwidth}{0.4pt}}

Sexuální násilí je takový nemorální sexuálně motivovaný čin nebo pokus o něj, při kterém jedna osoba sexuálním způsobem překročí osobní hranice druhé osoby. Děje se tak za použití násilí (fyzické převahy či převahy moci) ze strany pachatele, bez souhlasu či plného vědomí oběti a zcela nezávisle na vztahu pachatele a oběti. Podle definice Světové zdravotnické organizace (WHO) spadá pod sexuální násilí „jakékoliv sexuální jednání zahrnující pokusy o dosažení sexuálního styku, nežádoucí sexuální poznámky a návrhy, činy směřující k obchodování s lidmi či jinak namířené proti sexualitě jedince, které využívají nátlak“.
Sexuální násilí je skrytým a závažným zdravotním i sociálním problémem a týká se nemalé části populace. Je považováno za jeden z nejčastějších a nejrozšířenějších případů porušení lidských práv s nejvíce traumatizujícími následky pro oběti. V České republice je sexuální násilí posuzováno jako trestný čin proti lidské důstojnosti v sexuální oblasti (hlava III trestního zákoníku).

\hspace*{\fill} \emph{zdroj: wikipedia.org}


\end{document}
